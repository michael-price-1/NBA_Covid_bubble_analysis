% Created by Bonita Graham
% Last update: December 2019 By Kestutis Bendinskas

% Authors: 
% Please do not make changes to the preamble until after the solid line of %s.

\documentclass[10pt]{article}
\usepackage[explicit]{titlesec}
\setlength{\parindent}{0pt}
\setlength{\parskip}{1em}
\usepackage{hyphenat}
\usepackage{ragged2e}
\RaggedRight

% These commands change the font. If you do not have Garamond on your computer, you will need to install it.
\usepackage{garamondx}
\usepackage[T1]{fontenc}
\usepackage{amsmath, amsthm}
\usepackage{graphicx}
\usepackage{hyperref}
\usepackage{xcolor}
\newcommand{\jy}[1]{\textcolor{orange}{JY: (#1)}}
\usepackage{booktabs}

% This adjusts the underline to be in keeping with word processors.
\usepackage{soul}
\setul{.6pt}{.4pt}


% The following sets margins to 1 in. on top and bottom and .75 in on left and right, and remove page numbers.
\usepackage{geometry}
\geometry{vmargin={1in,1in}, hmargin={.75in, .75in}}
\usepackage{fancyhdr}
\pagestyle{fancy}
\pagenumbering{gobble}
\renewcommand{\headrulewidth}{0.0pt}
\renewcommand{\footrulewidth}{0.0pt}

% These Commands create the label style for tables, figures and equations.
\usepackage[labelfont={footnotesize,bf} , textfont=footnotesize]{caption}
\captionsetup{labelformat=simple, labelsep=period}
\newcommand\num{\addtocounter{equation}{1}\tag{\theequation}}
\renewcommand{\theequation}{\arabic{equation}}
\makeatletter
\renewcommand\tagform@[1]{\maketag@@@ {\ignorespaces {\footnotesize{\textbf{Equation}}} #1.\unskip \@@italiccorr }}
\makeatother
\setlength{\intextsep}{10pt}
\setlength{\abovecaptionskip}{2pt}
\setlength{\belowcaptionskip}{-10pt}

\renewcommand{\textfraction}{0.10}
\renewcommand{\topfraction}{0.85}
\renewcommand{\bottomfraction}{0.85}
\renewcommand{\floatpagefraction}{0.90}

% These commands set the paragraph and line spacing
\titleformat{\section}
  {\normalfont}{\thesection}{1em}{\MakeUppercase{\textbf{#1}}}
\titlespacing\section{0pt}{0pt}{-10pt}
\titleformat{\subsection}
  {\normalfont}{\thesubsection}{1em}{\textit{#1}}
\titlespacing\subsection{0pt}{0pt}{-8pt}
\renewcommand{\baselinestretch}{1.15}

% This designs the title display style for the maketitle command
\makeatletter
\newcommand\sixteen{\@setfontsize\sixteen{17pt}{6}}
\renewcommand{\maketitle}{\bgroup\setlength{\parindent}{0pt}
\begin{flushleft}
\sixteen\bfseries \@title
\medskip
\end{flushleft}
\textit{\@author}
\egroup}
\makeatother

% This styles the bibliography and citations.
%\usepackage[biblabel]{cite}
\usepackage[sort&compress]{natbib}
\setlength\bibindent{2em}
\makeatletter
\renewcommand\@biblabel[1]{\textbf{#1.}\hfill}
\makeatother
\renewcommand{\citenumfont}[1]{\textbf{#1}}
\bibpunct{}{}{,~}{s}{,}{,}
\setlength{\bibsep}{0pt plus 0.3ex}


\usepackage{float}

%%%%%%%%%%%%%%%%%%%%%%%%%%%%%%%%%%%%%%%%%%%%%%%%%

% Authors: Add additional packages and new commands here.  
% Limit your use of new commands and special formatting.

% Place your title below. Use Title Capitalization.
\title{The Effects of the NBA COVID Bubble on the NBA Playoffs: A Case Study for Home-Court Advantage}

% Add author information below. Communicating author is indicated by an asterisk, the affiliation is shown by superscripted lower case letter if several affiliations need to be noted.
\author{
Michael Price*, Jun Yan \\ \medskip 
Department of Statistics, University of Connecticut, Storrs, CT  \\  \medskip 
Students: michael.price@uconn.edu* \\
Mentor: jun.yan@uconn.edu 
}

\pagestyle{empty}

\begin{document}

% Makes the title and author information appear.
\vspace*{.01 in}
\maketitle
\vspace{.12 in}

% Abstracts are required.
\section*{abstract}
The 2020 NBA playoffs were played inside of a bubble at Disney World because of
the COVID-19 pandemic. This meant that there were no fans in attendance,
games were played on neutral courts and no traveling for teams. In theory, these conditions should
remove home-court advantage from the games. This setting generated
discussion and concern, as analysts and fans debated the possible effects it may have on the
outcome of games. Home-court advantage has historically played an influential
role in NBA playoff series outcomes. The 2020 playoffs provided a unique
opportunity to study the effects of the bubble and home-court
advantage by comparing the 2020 season with the seasons in the past.
While many factors contribute to the outcome of games, points scored
is the deciding factor of who wins. Thus, scoring is the primary focus
of this study. The specific measures of interest are team scoring totals and
team shooting percentage on two-pointers, three-pointers, and free throws.
Comparing these measures for home teams and away teams in 2020 vs.~2017-2019
shows that the 2020 playoffs favored away teams more than usual, particularly
with two-point shooting and total scoring.

% Keywords are required.
\section*{keywords} 
NBA; NBA Covid; NBA Bubble; HomeCourt Advantage;

\vspace{.12 in}

% Start the main part of the manuscript here.
% Comment out section headings if inappropriate to your discipline.
% If you add additional section or subsection headings, use an asterisk * to avoid numbering. 


\section{Introduction}

Home-court advantage is often discussed in sports circles as a contributing
factor to the outcome of games. It is well-known that the home team typically
benefits from some competitive edge while playing at their home court, resulting
in a better chance of winning. Thus, the NBA playing the 2020 playoffs in a
bubble due to the COVID-19 pandemic
brought a great deal of concern for fans, teams, journalists, and others. The bubble
environment would not be able to replicate many of the factors that normal home-court
advantage relies on. Among the biggest is the effect from the crowd. The home crowd has
the ability to energize home teams, help control momentum in favor of the home team and 
create a chaotic and stressful environment for away teams to play in. There is no way
to replicate these effects in the bubble conditions. Also, there are geographic factors
that are lost, like altitude and time-zone effects. There's evidence across all sports
that teams in high altitude regions can rely on away teams coming in and struggling with
the lower oxygen levels at a higher altitude, which the home teams are accustomed to playing
with \citet{Lopez}. Examples of high altitude teams in the NBA would be the Utah Jazz
and Denver Nuggets, both of whom participated in the 2020 NBA bubble. Also, referees 
are known to be subject to favoring home teams. This is due in large part to pressure
from fans, which is no longer in play in the bubble. Refs may still have biases, but 
there's nothing to sway that bias towards the home team like normal. Any positive
effects of the home team playing in the arena they're most comfortable in and
living in the comfort of their own home are also lost. Travel is also normally 
discussed as a factor hurting away teams, but in the context of the playoffs
being played in the bubble, losing this may not be significant. In the playoffs,
many times (aside from the first game of the series when the away team travels), 
both teams often travel at the same time, since they're both going back and forth
between the two cities to complete the series.
 
 \citet{Aschburner} discusses the anticipated effects, sharing
concerns from former players, coaches and other experts about
the potential effects of removing home-court advantage. Aschburner notes that the
NBA did make attempts to recreate the effects by putting the ``home'' team logo on
the court and allowing the ``home'' team to play crowd noise and music, but most
people doubted these small attempts would recreate a true playoff atmosphere.
During the 2020 NBA playoffs, home teams only won about 48.2\% of the games. This
is lower than normal, which Aschburner claims usually floats around 60\%. This
shift in the home team winning percentage surely indicates the opportunity for
thorough investigation.

So, what happened? Did the home teams fail to perform up to normal standards
without the help of home-court advantage? Were away teams able to rise to
the occasion and perform better not having to deal with the headache of going
on the road? We seek to answer the questions using scoring totals and shooting
percentages as indicators of team performance. This will deepen understanding
of how home-court advantage affects home and away teams in the NBA.

Our study is quite different from earlier NBA home-court advantage studies.
By using the neutral site games of 2020, we will get to compare home and away
performance to a control. Typically, studies just compare home vs away
performance. These studies do not separate the effects of home-court advantage
into the specific effect on the home team and the specific effect on the away
team. They show that home teams outperform away teams, but not if this is a
result of home teams overperforming or away teams underperforming because of
home-court advantage. Some of these studies are reviewed in greater detail in Section 2.

We will compare home team performance in 2020 at a neutral site with
no fans vs.~2017--19 playoffs with fans. Likewise, away team performance in 2020
at a neutral site with no fans vs.~2017--19 playoffs with fans. By comparing home
teams in 2020 to home teams in 2017--19 and away teams in 2020 to away teams in
2017--19, we add a new perspective to the field of research. This will allow for
a more accurate understanding of the effects of home-court advantage on home and
away teams in the NBA. We will not only see that home-court advantage helps home
teams outperform away teams, but also separate the
effects of home-court advantage on home teams' and away teams' performance individually.

Nine hypotheses were tested to understand the differences in 2020 vs.~earlier
years. First, whether or not the difference between home win percentage in 2020
and 2017--19 is zero. This difference is found to be statistically
significant from zero. Then we assess for differences in home
scoring in 2020 vs 2017-2019. Similarly, we can do the same test, but for
differences in away scoring in 2020 vs 2017-2019. Also, differences in team
shooting (for two-pointers, three-pointers, and free throws) from 2020 vs 2017-2019
for both home and away teams. The results from these tests bring a new perspective
to the understanding of how home-court advantage impacts games by altering the performance
of the home and away teams.


\section{Literature Review}

\jy{add some references from google scholar search ``home team
  advantage nba''; we have only 10 references now.}

There is voluminous literature on the effects of home-court advantage.
Many NBA home-court advantage studies analyze the effects by studying shooting
percentages. \citet{Kotecki} reported significant evidence of home-court
advantage by comparing field goal
percentage, free throw percentage and points scored in home vs.~away teams.
He found all of these measures showed evidence that
home-court advantage helps home teams play better. \citet{Cao} studied the effects
of pressure on performance in the NBA. Using free throws as their measure of interest, 
they tested whether home
fans could distract and put pressure on opposing players to make free
throws. However, they did not find significant evidence
that home status has a substantial impact on missing from the free throw line. 
\citet{Harris} used two-point shots, three-point shots and free
throws as measures of interest to study home-court advantage. Two-point
shots were found to be the strongest predictor of home-court advantage. They
suggested that home teams should try to shoot more two-point shots and force
their opponent to take more two-point shot attempts. This help home teams have a
greater control of the game play and help maximize the benefits 
home-court advantage.

Some studies focus less on shooting and more on other metrics.
For example, \citet{Greer} focused on the influence of spectator booing on
home-court advantage in basketball. The three methods of performance used
in this study were scoring, violations, and turnovers. This study was conducted
using the men's basketball programs at two large universities. The study finds that
social support, like booing, is an important contributor to home-court advantage.
Greer explains, whether the influence is greater on visiting team performance or
referee calls is less clear. However, the data does seem to lean slightly in
favor of affecting visiting team performance. Another study focused on scoring
was conducted by \cite{Jones}. \cite{Jones} analyzed scoring patterns across each of the 
4 quarters to analyze the effects of home-court advantage over the course of a game.
He found that typically 2/3 of the benefits of home-court advantage are received in 
the first quarter with the remaining 1/3 slowly accumulating over the 
final 3 quarters. This implies home-court advantage is most effective when teams 
use it to build a lead they then just have to maintain for the remainder of the game.
If they are losing after the first quarter, there is a sharp drop in win probability. 

There are also surveys on the factors contributing to home-court advantage.
\citet{Carron1992} gave four main game location factors for home and away
teams, namely, the crowd factor, which is the impact of fans cheering; learning
factors, which is an advantage from home teams from playing at
a familiar venue; travel factors, the idea that away teams may face
fatigue and jet lag from traveling; and, rule factors, which says that home teams
may benefit from some advantages in rules and officiating. They acknowledge that
these factors would all be removed if games were played at a neutral site, even
if one team was designated as ``home team''. This study was reviewed a decade later
by \citet{Carron2005}. The 2005 review goes over the new findings from studies
about the significance of these four game location factors. Since 1992, they have found
that results on these four factors are mixed. However, there is some evidence
supporting crowd and travel factors impact games across all major sports. There is less
evidence suggesting learning and rule factors impact across the various collegiate 
and professional sports. The NBA is not a league which has rules that
may favor the home team, like batting last in the MLB, but these rule
factors also account for referee bias which may impact the NBA.  One
interesting finding cited by \citet{Carron2005} is that the absence of crowds results
in overall performance increases. Another study by \citet{Price} is able to find 
some evidence of referee bias. They focus on two measures, DTOs and NTOs. DTOs, 
discretionary turnovers, are defined as turnovers always caused by the ref blowing
the whistle while the ball is in play. NTOs, non-discretionary turnovers, are determined 
directly by players with no ref whistle, or the ball going out of bounds. The use
these to test ref bias by checking how variables, like home vs. away team, affect
DTOs relative to NTOs. They found evidence that a home bias does exist. In fact, home 
bias increases both during the playoffs and in games with higher attendance. This is
crucial to the NBA bubble, which consists of playoff games with no attendance. Lastly, 
as discussed in the introduction, \citet{Lopez}
find evidence that geographic factors like altitude may influence and strengthen home-court
advantage for teams in high altitude regions.

There are a few examples of natural experiments in basketball. \citet{Harville} studied the
effect of home-court advantage using the 1991-1992 college basketball season.
Unlike the NBA, it is not uncommon to have a few games played at neutral sites
during the college basketball season. This allowed them to construct two samples,
one of home teams and one of neutral teams. They formulated their study in a
regression predicting the expected difference in score for home teams.
They set up their study to find if the home teams won games by more points when
they had home-court advantage vs.~when playing on a neutral court. This study
concluded with evidence supporting home-court advantage. Also, \cite{Boudreaux}
is able to construct a natural experiment using the Los Angeles Lakers and Los
Angeles Clippers. Since these two teams share a home stadium, many factors like 
travel and familiarity are nullified. However, the designated home team has larger
crowd support due to attendance from their season ticket holders. When they
single out the effect of having a sympathetic crowd, \cite{Boudreaux} estimate crowd 
effects increase the chance of winning 21--22.8 percentage points.


\section{Data}

Data were collected from the official NBA website. The main variables of interest
are whether or not the home team won, scoring totals for home and away teams, and
shooting percentages for home and away teams on two-pointers, three-pointers, and
free throws. These variables were very popular and frequently used in the related
literature discussed earlier. While many other measures could be used for examining
the outcome of the game and team performance, scoring seemed to be the most
important because the winner of a game is determined by who scores more points. Furthermore, 
the three types of shots are a
natural discussion point in basketball, as they're all
important and directly impact scoring totals. Free throws can often determine the winner of close
games, especially when losing teams are forced to foul the winning team to
stop the clock and hope for some missed free throws. The three
point shot has grown very prominent in basketball. The entire NBA has increased their
volume of three-point shot attempts in response to the
recent success of the Golden State Warriors and Stephen Curry. Due to this
fact, there is a growing consensus that the three-point shot is crucial because of 
the efficient scoring and floor spacing it provides offenses. 
However, two-point shots are not to be overlooked and may
actually be most important, especially for home teams, according
to \citet{Harris}. In fact, the 2020 NBA champion Los Angeles Lakers actually led the 
league in two-point shooting percentage throughout the playoffs and were bottom five
in three-point shooting percentage.

The data was collected on a game by game basis. This gave us two observations
for each variable per game played, one observation for each team(home and away).
There were 83 games played in the 2020 playoffs, giving 83 observations
for each variable in 2020 for both the home and away teams (166 observations total).
Likewise, there were 243 games played over 2017-2019, giving 243 observations of each variable
for both home and away teams over 2017-2020 (486 observations total).
 There can only be
one winner and one loser, making the outcome a binary variable, with
one indicating a win and zero indicating a loss.

Home-court advantage is the basic idea that the home team is more likely to win given 
they benefit from positive effects of a few factors we discussed earlier.
So laying a foundation of typical home-court advantage is crucial. Before
focusing on the 2017 to 2020 playoffs, we can take a quick look at home team win
percentages since 2010. Notice in Figure~\ref{fig:Fig1}, the 10 years before 2020, the home team
winning percentage ranged from around 0.56 to 0.7 and never dipped below 0.5. The
2020 bubble broke this historic pattern, dipping down below 0.5. Foreshadowing
the confirmation of the expectation that the effect of home-court advantage was removed
in the 2020 playoffs.

\begin{figure}[tbp]
  \centering
  \includegraphics{Fig1}
  \caption{Winning percentage of NBA home teams in the playoffs since
    2010, the green line denotes .500.}
  \label{fig:Fig1}
\end{figure}

Moving on to the main focus of the study, comparing 2020 to 2017-2019.
Figure~\ref{fig:Fig2} shows the histograms of the home (green) and
away scoring (red) for 2020 vs.~2017-2019. All histograms are
fairly bell shaped, which is important for statistical tests
designed for normally distributed data. There appears to be little
difference between the 2020 and 2017-2019 for home scoring. For
away scoring, a more pronounced shift to the right in 2020 is observed
compared to 2017-2019.


\jy{remake the figure~\ref{fig:Fig2}: make each panel aspect ratio 3:4; same for the
  next figures.}

\begin{figure}[tbp]
  \centering
  \includegraphics{Fig2}
  \caption{Histograms of home (blue) and away (red) scoring
    for 2020 (bottom) and 2017-2019 (top).}
  \label{fig:Fig2}
\end{figure}

Our second target of inference is shooting percentage for home and away teams.
Figure~\ref{fig:Fig3} shows home shooting for two-pointers, three-pointers and
free throws for 2020 (top) vs.~2017--19 (bottom). The histograms appear to be fairly
similarly distributed between 2020 and 2017--19. Likewise, Figure~\ref{fig:Fig4},
shows the same percentages except for away teams. It appears that the
two-point shooting percentage for away teams has a small shift to the
right in 2020 relative to 2017-2019.




\begin{figure}[tbp]
  \centering
  \includegraphics{Fig3}
  \caption{Histograms of home shooting percentages for two-
    pointers, three-pointers and free throws for 2020 (top) vs.~2017--19
    (bottom).}
  \label{fig:Fig3}
\end{figure}




\begin{figure}[tbp]
  \centering
  \includegraphics{Fig4}
  \caption{Histograms of away shooting percentages for two-pointers,
    three-pointers and free throws for 2020 (top) vs.~2017--19
    (bottom).}
  \label{fig:Fig4}  
\end{figure}

\section{Methods}

The 2020 bubble provides a new and exciting opportunity to study home-court
advantage for the NBA. Unlike college basketball, aside from a few
exhibition/preseason games, the NBA always has a home and away team. So, for
the first time in NBA history, the bubble allows NBA home and away performance
to be compared against a control/neutral field. The NBA bubble
removed many, if not all, factors impacting home-court advantage. The NBA 
bubble featured 8 seeding games for each team, then a
standard playoff format. The focus of this study was on the play during the playoff
games, since it followed the standard playoff format and could be compared
back to other playoffs. For this study, the 2020 playoffs were compared against
the three previous playoffs collectively. To control
for the changing play style of the NBA, we limit the study to 2020 vs 2017-2019
due to the faster pace of play and more common use of the three-point shot
in modern basketball. If we used data from say 10 years ago, or earlier, observed
differences may not be from effects of the NBA bubble, but rather from the effects
of drastic changes in the style of play between the seasons. However, basketball
evolves slow enough that we can reasonably assume 2017-2019 are at least very
close in pace and playing style to 2020.

Comparisons between 2020 and 2017--19 home and away teams were made on home team
winning percentage, total team scoring and two-point, three-point and free throw
shooting. Comparing the differences in these metrics for home and away teams in
2020 vs previous years will provide valuable insights to the understanding of
home-court advantage. We can see how going on the road may negatively impact
away performance and how playing at home may positively impact home performance.
If there are differences in scoring for home or away teams, the differences can
be used to show how home-court advantage affects the overall performance of home
and away teams. While testing for differences in shooting will provide added
context for how home-court advantage specifically affects performance.
Shooting percentages are not the only possible metrics affected by home-court
advantage, but they are the most obvious and likely most important one.

We formulate the following nine specific research questions to test
the effects of the COVID bubble on the 2020 NBA playoffs:

\begin{enumerate}
\def\labelenumi{\arabic{enumi}.}
\item
  Is the home team winning percentage in 2020 different than that it was in 2017-2019?
\item
  Is the average home team scoring different in 2020 than it was over 2017-2019?
\item
  Is the average away team scoring different in 2020 than it was over 2017-2019?
\item
  Are home teams making two-pointers at the same rate in 2020 as 2017-2019?
\item
  Are home teams making three-pointers at the same rate in 2020 as 2017-2019?
\item
  Are home teams making freethrows at the same rate in 2020 as 2017-2019?
\item
  Are away teams making two-pointers at the same rate in 2020 as 2017-2019?
\item
  Are away teams making three-pointers at the same rate in 2020 as 2017-2019?
\item
  Are away teams making free throws at the same rate in 2020 as 2017-2019?
\end{enumerate}

All nine questions can be approached by a standard two-sample
comparison with the \(z\)-test. The \(z\)-test statistic follows a
standard normal distribution, which is a good approximation based on
the central limit theorem given the sample size in this application.

We also conducted nonparametric tests that are distribution free to
confirm the results from the \(z\)-test. For question 1, we used
Fisher's exact test for a contingency table which summarizes the wins
and losses of the home team in the 83 games in 2020 and the 243 games
in 2017-2019. For all other eight questions, the data are the scores
or shooting percentages from the 83 games in 2020 and the 243 games in
2017-2019. We used Wilcoxan's rank-sum test.

All three tests, namely the \(z\)-test, Fisher's exact test, and
Wilcoxan's rank-sum test, were performed using R \citep{R}.

Regression was also considered, but it did not seem to add any additional 
useful insight to answering the questions presented. Regression is a popular
tool in home-court advantage studies, but as mentioned previously, this
study is very different from past studies. Rather than trying to prove the
existence of home-court advantage, like many previous studies, this paper
was more interested in generating a clearer understanding of how home-court
advantage affects each team. The most effective way to do that is hypothesis
testing to compare performance in the neutral bubble to previous years with
normal game conditions for both the home and away teams. Another potential
benefit of regression would have been the ability to control for factors
like the strength of opponent. However, we felt that the strength of team was
already well enough controlled for by the fact that we focused on the
playoffs. The playoffs only include the strongest 16 of 30 teams, then
continue to remove the less competitive teams, so that the talent disparity
between teams is much smaller than regular season play, where controlling for
opponent strength is more likely to be important.

\section{Results}

\begin{table}[tbp]
  \caption{The results from the 9 tests(* denotes significant p-value at alpha=0.05)
  \\*Note: Home and Away scoring totals are divided by 100}
  \label{tab:table}
\centering
\begin{tabular}[t]{lccllll}
  \toprule
  & 2020 & 2017--19 & \multicolumn{2}{c}{P-value} & \multicolumn{2}{c}{Adjusted P-value}\\
\cmidrule(lr){4-5}\cmidrule(lr){6-7}
  &          &                & \(Z\)-test & Wilcoxon & \(Z\)-test & Wilcoxon\\
  \midrule
  Home Win & 0.482 & 0.613 & 0.0497* & 0.0400* & 0.4473 & 0.3598\\
  \midrule
Home Scoring & 1.101 & 1.081 & 0.2321 & 0.2985 & 1.0000 & 1.0000\\
\midrule
Away Scoring & 1.091 & 1.040 & 0.0008* & 0.0004* & 0.0075* & 0.0035*\\
\midrule
Home 2P & 0.523 & 0.515 & 0.4335 & 0.5719 & 1.0000 & 1.0000\\
\midrule
Home 3P & 0.363 & 0.357 & 0.5733 & 0.8852 & 1.0000 & 1.0000\\
\midrule
Home FT & 0.793 & 0.774 & 0.0692 & 0.0496* & 0.6228 & 0.4464\\
\midrule
Away 2P & 0.536 & 0.504 & 0.0003* & 0.0003* & 0.0030 & 0.0023\\
\midrule
Away 3P & 0.357 & 0.346 & 0.3256 & 0.3081 & 1.0000 & 1.0000\\
\midrule
Away FT & 0.783 & 0.777 & 0.6601 & 0.8370 & 1.0000 & 1.0000\\
  \bottomrule
\end{tabular}
\end{table}

Starting from the top,
Table~\ref{tab:table} summarizes p-values of the nine hypotheses for both
\(z\)-tests and Wilcoxon tests. The p-values are all fairly similar for both
tests giving strong confidence in the conclusions drawn. Additionally, the adjusted
p-values, calculated using a Bonferroni correction, and point estimates
for each sample are provided.

First, we see a statistically significant change
in home win percentage in 2020 from 2017--19, with a p-value of 0.0497 for the
\(z\)-test and 0.0400 for Fisher's exact test. The 95\% confidence
interval (CI) of \((-0.255, -0.008)\)
confirms our belief that home-court advantage was lost in the 2020 NBA 
playoffs. However, after accounting for
multiple tests using the Bonferroni correction, the p-values for both
tests are no longer significant. So, we may only cautiously say there is 
evidence that home-court advantage was
not a factor in 2020.

Home team performance did not seem to be negatively impacted by losing home-court
advantage like expected. Home scoring, two-point and three-point
shooting all show no significant difference, on average, between 2020 vs.
2017--19 based on p-values from both tests. However, the Wilcoxon test and \(z\)-test have conflicting results
for free throws. The \(z\)-test p-value of 0.0692 indicates no significant difference, while
the Wilcoxon test p-value of 0.0469 indicates a difference at the 5\% significance level. Since
the p-value of the Wilcoxon test is so close to significance level and neither p-value is significant
after a Bonferonni correction for multiple tests, this difference is likely not very meaningful.
There appears to be no strong evidence suggesting home teams played at
a lower level in 2020 than they did in previous years when they had home-court advantage.

Away teams saw more of an impact than home teams. For starters, there is a
significant increase in mean points per game, indicated by p-value of 0.0008 for \(z\)-test and
0.0004 for Wilcoxon. It is important to note both p-values also remain significant after
a Bonferonni correction giving strong indication of significance. The
average difference in points was estimated to
be about 5 points, with 95\% CI \((2.083, 7.988)\). Likewise, the away team two-point
shooting efficiency increased significantly based on p-value of 0.0003
for both the \(z\)-test and Wilcoxon test. Again, both p-values remain significant after Bonferroni correction.
The average difference was estimated to be about 0.03, with 95\% CI \((0.015, 0.050)\). However,
unlike two-point shooting, away teams did not see a statistically significant difference in three-point and free
throw shooting. Overall, away teams have evidence of change in performance in
the bubble. The away teams seemed to perform better than they would under normal
conditions as a visiting team.

\section{Discussion}

Generally, it seemed that away teams fared better in the 2020 NBA playoff bubble
than previous years on the road. Starting from the dip in home winning percentage to
below 0.482, it is clear that something was different. Although the difference
was not significant after a Bonferroni correction, it is still informative to 
consider and understand that home teams seemed to struggle to win compared to
normal conditions. Compared to \citet{Kotecki}, who finds home teams consistently 
have a significantly better record than away teams, boasting about a 60.5\% win 
percentage in his sample, the 48.2\% home winning percentage of 2020 home teams 
is quite a shift. In this study, home teams did not appear to benefit from the 
usual advantages provided by being the home team.

Away team average scoring did increase by a statistically significant amount.
This goes hand in hand with our intuition and conclusion about the home winning 
percentage decreasing. If away teams are scoring significantly more and home 
teams are not, then we expect to see away teams winning a larger number of games. 
This may give more reason to believe the conclusion that there was a significant 
decrease in home winning percentage in 2020, despite failing to be significant after 
the Bonferroni correction. Only away team scoring being significantly impacted by 
playing on a neutral court and not the home scoring indicates that home-court 
advantage stems mainly from adverse effects on the visiting team.

At least some of that improvement from away teams came from significantly higher
two-point efficiency. This corresponds with the conclusion from \citet{Harris},
where they found home teams are best suited to capitalize on advantages from two-point
shots. Normally, by shooting more two-pointers themselves and forcing away teams to
shoot more two-pointers, the home team benefits most from effects of
home-court advantage. However, with away teams significantly improving two-point
shooting in the bubble, this strategy was no longer viable and home-court
advantage disappeared.

To get a bit more detailed, we saw an estimated 3\% increase in away two-point
shooting percentage. Likewise, NBA teams shot
an average of about 53 two-point shots per game, in our data, leading to
a three-point increase in total scoring. Even this
small swing in scoring for away teams can make a big difference in outcome of
games. Around 1/3 of NBA games end by a decision of six points or less, these
are the games where the three points matter most. Suppose there are 30 seconds
left and the away team is losing. If the away team has the ball, only down three, 
they can take their time finding a good shot which can
still either be a two or three given the score and remaining time. In the context 
of the NBA bubble, this crucial possession also has the benefit of the away team
getting to work in silence without the jeers from enemy fans. However, in the
same situation, while being down six, the offense must rush to get a quick shot
off which probably has to be a three-pointer. The three point swing creates a much more desperate
circumstance that is less likely to have a positive outcome. Even in the positive
outcome case you're still going to be down three and likely have to foul the
opposing team after you score. 
Alternatively, if the away team is playing defense with 30 seconds left, only down 
three, you can play regular defense without fouling, get a stop, then draw up a play
to score a three-pointer with the remaining six seconds. In the same scenario, down 
six, you have no choice, but to foul. The free throws will likely put you down seven
or eight points, but the alternative is letting the opponent run the 24 second shot 
clock down wasting time. Similarly, the away team may even be in a situation where 
they're leading by the three additional points, which for the same logic as explained
above puts them in far greater control of the game. Clearly, the three point swing 
can drastically change probability of winning for a team down the stretch of a game. 
Since, home teams picked up an additional two-points per game on average vs. five for 
away teams, the three additional points from two-point shooting appear to be the driving
factor of away teams closing the gap with home teams.

There are a few possibilities that might create this increase in two-point shooting
percentage. One could be the possible removal of officiating bias in favor of the 
home team during the bubble. As we saw in \cite{Price}, home bias is normally
stronger in high attendance games. Thus, with no attendance at games, it's reasonable
to say officiating bias was smaller than normal. Away teams in the bubble may have benefited from more 
fouls called on drives to the basket. On drives to the basket there can often be a 
lot of contact with no clear foul, these calls are then up to the ref's discretion.
Typically, refs may be more reluctant to blow the whistle against the home team, in
front of their crowd, on this type of play. The NBA only records an attempted shot 
when a foul is called on the shot if the shot is made. So, when away teams are 
fouled, but it is not called because it's often at the ref's discretion to make 
the call, they'll likely miss and be credited the miss. However, in the bubble, if
fouls are called more fairly, the missed shots from uncalled fouls are removed and 
two-point shooting would increase. This wouldn't affect three-point shooting because
those fouls are clearer and less up to the ref's discretion. Another possibility,
without the crowd noise inhibiting their offense, away teams were able to more 
easily run their offensive sets that generate easy two-point looks at the rim. 
Also, away teams being generally more confident without opposing fans present, 
may have been more inclined to attack the basket and get an easy look close to 
the rim. It's hard to say for sure what causes the increase in away two-point shooting,
but these are all possibilities. It may also be a combination of all of these. 


Separating the effects of the home-court advantage into home effects and away
effects allowed for some interesting new insights. Previously, we knew that on
average home teams outperformed away teams. It was less clear whether it was
from positive effects on the home team or negative effects on the road team or
perhaps a bit of both. The biggest takeaway from this study is the main source
of home-court-advantage is the negative effects playing on the road away teams
face. In 2020 there wasn't any evidence of regression for home team performance,
based on the performance measures used, despite being stripped of home-court
advantage. Yet, home teams lost about 12\% more of games in the 2020 playoffs
than the typical average. This was because of the improvement of away teams.
No longer having to face the struggle of traveling, pressure from opposing fans,
or playing on an unfamiliar court, teams saw an improvement in their play and an
increase in winning. The improvement of away teams confirms a proposition from
\citet{Greer} that the positive social impact of crowds benefiting
home teams may be a result of inhibiting away teams.

It's worth noting that much of this paper discusses the effects from an
offensive point of view, focusing on a team's ability to score. However, one of 
the main factors that affects an offense's ability to score is the opposing
defense. It's possible that home-court advantage mainly functions as an extra
defender for home teams, which is how it negatively impacts away scoring. There
is not much home fans can do to help their team on offense, except being quiet
so players can easily communicate and focus. However, when on defense, fans can
act as a "sixth defender" supporting their team by cheering loudly and making it
harder on the offense. Also, as seen in the study \cite{Price}, fans can 
create turnovers by pressuring officials to make calls in their favor.  
This would explain how away teams were able to improve
offensively in the bubble. Home teams did not have the regular help from the
noise of fans that hinders the away team's offensive efforts. This would also
explain why home team offense didn't show significant change in the bubble, the
in-game offensive environment was largely unaltered for home teams since the
bubble provided the same quiet environment they're used to at home while on
offense. I still believe home-court advantage is better categorized as a
negative effect on away team offense, since it's more likely
noise from fans lowers level of play of away offense rather than raises the level
of play of home defense. This will be further discussed below.

An interesting finding is all shooting and scoring numbers for both home and
away teams did make at least small increases. Although, these increases
were not all significant these increases are exactly what is reported in
\citet{Carron2005} when they explain how evidence suggests that teams perform
better with the absence of fans. This is important because it coincides with our
conclusion that home-court advantage mostly plays into games by negatively impacting 
away teams by acting as an extra defender. If fans cause overall performance to drop,
then home-court advantage must come from a bigger drop in away performance than the 
drop in home performance. This is why away teams were able to close the gap with 
home teams with home-court advantage removed.

Future studies may want to use the 2020 NBA bubble and compare vs previous years
using other performance measures. For example, turnovers, steals, assist, rebounds,
and many more game statistics. There are plenty of other possibilities besides
just shooting efficiency to pick through looking for more possible sources of
added points for away teams. This will further help explain what is lost in the
performance of away teams when they travel to opposing arenas. This study is only
the beginning of possibilities for studies using the 2020 NBA bubble as a case study
for home-court advantage. Although, the study is limited by a one time
sample, it seems unlikely that these conditions will ever be
repeated. It may not be possible to have a follow-up study using the
same measures with a different sample. Otherwise, that type of study
could help strengthen the conclusion in this paper. Also, you could
answer more questions with a larger sample size. For example, testing
the strength of home-court advantage relative to specific teams, like the
Denver Nuggets and Utah Jazz who we know from other literature may
have a stronger home-court advantage. The sample size of home games
for the individual teams in the bubble was far too small to try and
address this question, the Jazz and Nuggets only had 3 and 9 home
games, respectively.


\bibliographystyle{chicago}
\bibliography{citations.bib}

% \section*{acknowledgements}
% The authors thank the University of Connecticut and the UConn Department of Statistics.
%Note correct LaTeX quotations above. Do not use the " symbol, but rather double ` followed by double '


% The About the Student Author section is NOT optional.  Write a paragraph about the student; see previous journal editions for examples.
% If there is more than one student author, you must move the comment below.
\section*{about the student author}
%\section*{about the student authors}
Michael Price graduated in May 2021 from the University of Connecticut
where he double-majored in Statistics and Economics and minored in
Data Analytics. He began his graduate studies at the University of
Delaware studying Applied Statistics in August 2021.


% The Press Summary section is NOT optional.  Write a paragraph describing the paper in a manner suitable for the press; see previous journal editions for examples.
\section*{press summary}
The purpose of this study was to examine the 2020 NBA playoffs, which
were played inside of a bubble in Disney World because of the COVID-19
pandemic. The hope was to generate new insights about the effects of
home-court advantage because the bubble created an unprecedented
neutral playing field for the NBA. For the first time the effects of
home-court advantage on home and away teams could be easily separated
and studied individually, typically studies can only compare them
relative to each other. This study is focused on team scoring totals
and team shooting percentage on two-pointers, three-pointers, and free
throws. Comparing these measures for home teams and away teams in 2020
vs. 2017--2019 shows that the 2020 playoffs favored away teams more
than usual, particularly with two-point shooting and total
scoring. The implication of these findings is home-court advantage
seems to be the result of negative effects on away team, not positive
effects on home-team.



\end{document}
