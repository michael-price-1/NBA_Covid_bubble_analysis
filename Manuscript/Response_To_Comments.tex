\documentclass[12pt]{article}
\usepackage[margin=1in]{geometry}

\usepackage{xr-hyper}
\RequirePackage[colorlinks,citecolor=blue,urlcolor=blue]{hyperref}
\usepackage{listings}
\usepackage{rotating, graphicx}
\usepackage{booktabs, natbib}

\usepackage{amsmath, enumerate, quoting, amssymb, subfigure, url}

\usepackage{color}
\newcommand{\red}[1]{\textcolor{red}{#1}}
\newcommand{\blue}[1]{\textcolor{blue}{#1}}
\newcommand{\gv}[1]{\textcolor{blue}{(GV: #1)}}
\DeclareMathOperator*{\argmin}{arg\,min}
\DeclareMathOperator*{\cov}{Cov}
\DeclareMathOperator*{\corr}{Corr}

\usepackage{xr}
\externaldocument{isgee}
\externaldocument{supp}

% remove left indentation in itemize
\usepackage{enumitem}
\setlist[itemize]{leftmargin=*}

\usepackage{xcolor}
\newcommand{\jy}[1]{\textcolor{red}{(JY: #1)}}

% graph path
\graphicspath{{./figs/}}

\quotingsetup{leftmargin = 0pt}

\newenvironment{comment}%
{\begin{quoting}\noindent\small\it\ignorespaces%
  }{\end{quoting}}




\begin{document}

\begin{center}
  {\Large\bf Response to the Comments on Manuscript}
\end{center}

\section*{Summary}

We thank the two referees for their insightful, constructive
comments. In particular, we are grateful to referee~1 for providing
many helpful, direct and concise comments using annotations. Also, to
referee~2, who took the time to provide detailed explanations for each
suggestion making it clear how and why the changes were important.

The manuscript has been revised to incorporate the comments from the
referees with the following most notable additions:

\begin{enumerate}
\item
  Additional commentary and thoughts on what caused away teams to
  be more affected by neutral courts more than home teams.
  \item
  A deeper discussion on the significance and impact of the shift 
  in two-point shooting.
\item
  Alterations to graphs and charts to better and more clearly describe
  the data and findings.
\end{enumerate}

Point-by-point responses are as follows with the original comments in \emph{italic}.


\section*{To Referee~1}

\begin{comment}
  I’d have the paper examined for grammar.  I think there are commas missing or 
  misplaced periodically. Couldn’t add that to the pdf effectively.
\end{comment}

 The paper was carefully reread and edited, line by line, in order to
 correct grammar errors and improve the overall flow of the paper.
 
% \begin{comment}
% Comments in the pdf. 
% \end{comment}

There were a series of brief comments provided on an annotated version of our
manuscript submission. This included many small comments for missing words or 
asking for a brief clarification about a point made. These comments were addressed
accordingly and were resolved. The more extensive comments/concerns are addressed 
individually below.

\begin{comment}
  Reference lines (mean/median) would be nice in the graphs to more clearly
  see the small shifts.
\end{comment}

The figures were remade with added reference line at the mean for all 
applicable histograms, in the Data section.
 
\begin{comment}
Use of adjusted p-values was appreciated, but those should be listed in 
the table.  I would also recommend using either the nonparametric (Wilcoxon)
or parametric (t-test) exclusively—no need for both.  I think the 
assumptions are satisfied for the parametric test.  Fisher’s exact test 
is appropriate
\end{comment}

 Columns were added to the results table, located at the beginning of the Results section,
 to accommodate including the adjusted p-values. As for including both the non-parametric
 and parametric tests, we think including both is justified. We felt it strengthens the 
 confidence in the results when the conclusions from each the non-parametric and parametric 
 tests agree with one another. While the assumptions appear to be met for parametric
 test, it does not hurt to show tests not dependent on assumptions also show
 significant results.
 
\begin{comment}
 Why are two pointers singled out in the data? Any insight on why 2P\% is 
 affected more? I would provide some insight. My hypothesis is officiating 
 bias as shown in many papers (less discretion on 3P fouls). 
\end{comment}

 We thank the reviewer for bringing this concern to our attention. We realize
 we missed discussing the significance of studying each type of shot individually,
 not just 2-pointers. Discussion on the importance of examining each of the three
 types of shot was added, including the 2-point shot, at the end of the first 
 paragraph in the Data section.

\begin{comment}
Comment a bit on these effect sizes. Does a 3\% decrease in 2P\% matter in a game?
That may only be 1 shot. While it's statistically significant, I wonder 
if it's practically significant.
\end{comment}

 A paragraph was added to the Discussion section(fourth paragraph) detailing the practical 
 significance of the 3\% change in 2-point shooting. In short, many NBA games 
 are decided by 6 points or less and in those games late-game strategy is 
 completely dependent exactly how many points one team is behind. Even a small
 shift in score can drastically shift probability of outcomes and strategy 
 for teams. The paper now goes into detail on how a 3-point swing in these 
 close games can completely change the game.

\begin{comment}
Would be interesting to examine if these home court advantages are 
more pronounced for specific teams.
\end{comment}

 We are appreciative of this suggestion for another possible topic of analysis. 
 Cross referencing this comment with the source provided by referee~2, we definitely 
 have reason to believe the shift in home-court advantage would be more pronounced 
 for specific teams. However, we didn't have the sample size to test for strength 
 home-court advantage in individual teams. The teams we'd expect to have the 
 most pronounced affect, the Denver Nuggets and Utah Jazz, only played 9 and 3
 home games, respectively, in the bubble. We did add some comments about this 
 at the end of the last paragraph in the discussion section which explains
 how this analysis would be possible if generating a larger sample were
 possible and what we'd expect to see.


\section*{Referee 2}

\subsection*{Major comments}

\begin{comment}
Good motivation of the problem and why we have additional useful information
to answer questions about
home advantage.
It would be good to mention here, or elsewhere, different (potential) 
sources of home advantage, and discuss
which may/may not still exist in the NBA bubble. For example,
\item
• Travel miles/time
\item
• Time zone changes
\item
• Days rest (back-to-back)
\item
• Altitude
\item
• Attendance/crowd
\item
• Referee bias
\item
• Being at home, familiar surroundings, sleeping in own bed
\item
This is partly in the literature review where you discuss Courneya and Carron’s article,
but I think you could touch on that there are different sources of home advantage 
somewhere in the introduction. There are, for example, no travel miles differences between the 
two teams and (I think) no back-to-back games, so homeadvantage from those probably doesn’t 
exist in the bubble. There may, for example, still be referee bias in
the bubble, though. Referee bias may only exist when there’s a home crowd, this
case the video monitors and the handful of attendees don’t affect referee. 
Or it might exist because referees know who the home team is.
\end{comment}

 We thank the reviewer for being appreciative and understanding of the 
 good motivations driving this paper. We agree that a deeper discussion 
 of why home-court advantage would not exist in the bubble would be beneficial 
 to setting up the paper. We added in a discussion of why all the different 
 factors play into creating home-court advantage and how each one may or may 
 not still exist in the bubble. This addition to the paper can be found in the first paragraph
 of the Introduction section. For the most part, these factors disappeared 
 in the context of the bubble, except travel/time and miles for reasons we
 discuss in the paper. 

\begin{comment}
Also, this paper https://arxiv.org/abs/1701.05976 finds that altitude matters
(Colorado and Utah teams have the highest home advantage in multiple sports),
so that could be added/discussed. So changes in home advantage in the bubble
could impact teams differently, and NBA teams Nuggets and Jazz could be hurt
more than other teams since they no longer benefit from an altitude advantage.
\end{comment}

 We are very grateful that referee~2 brought \citet{Lopez} to our attention. 
 It was added to Literature Review(at the end of the third paragraph) and
 discussed in the Introduction(first paragraph) when setting 
 up and explaining the factors that would be missing from normal home-court 
 advantage when playing in the NBA bubble. The most noteworthy point added 
 from this paper, as pointed out by referee~2, was the evidence supporting 
 geographic factors impact home-court advantage. Particularly, teams like 
 the Utah Jazz and Denver Nuggets are the NBA teams who would benefit from 
 the high-altitude effects. This is discussed in more detail in the manuscript.
  Also, a few other references were added to strengthen the paper.
  One, \cite{Price} discuss empirical evidence in favor of referee bias towards home teams.
  

 \jy{Check the references of Lopez's paper to see if we can cite
   anything there; this is also a place to mention the additional
   references you found. }

\begin{comment}
In your plots with histograms, it would be useful to have a vertical 
line showing the mean to visually compare
the two distributions.

It seems like you made this using facet\_grid(shot.type ~ season) or
something similar. I think it would be easy to add vertical lines fairly 
easily by using group\_by and summarise to make a data frame df.means
like this

season shot.type mean.perc

2020 2p xx.x

2020 3p xx.x

2020 ft xx.x

2017-2019 2p xx.x

2017-2019 3p xx.x

2017-2019 ft xx.x

and then adding
geom\_vline(data=df.means, aes(x=mean.perc)) +
somewhere in your ggplot code.

If the columns season and shot type are the same as in your other data
set, and the same names used in facet\_grid, a different vertical line
should show up for each season and shot type.
\end{comment}

 While this was not the exact coding solution used to add reference lines
 to the histograms,it was greatly appreciated and was the perfect stepping 
 stone to finding the code solution. As mentioned earlier, the code was
 added to include a reference line at the mean on each histogram we provided,
 within the Data section of the manuscript.

 
\begin{comment}
If you use a regression, you could take into account strength of team,
strength of opponent, and home advantage simultaneously. This could 
be done with any of the metrics you considered. For example,
Score ~ 1 + HomeTeamOnOffenseIndicator + OffensiveTeam + DefensiveTeam
Did you consider a regression? Why would it, or would it not, be useful? 
If strength of opponent (in the example above, the DefensiveTeam) doesn’t
vary much from team to team (in other words, if there is a
balanced schedule), then it might not be that helpful to account for 
strength of opponent. If it does vary a lot, then it might help.
In a typical regular season, the NBA schedule is pretty balanced. 
Is the schedule a balance schedule in the bubble?
Would accounting for team strength change the home advantage estimates? If so, 
in what situations would it increase home advantage estimates? Decrease?
You don’t necessarily have to do all of this analysis, but you should at least 
discuss somewhere why you didn’t use a regression, and justify your decision.
I’m not saying that a regression is necessarily better, and the answers to 
the above questions might suggest it wouldn’t help. But a lot of people
would think of estimating home advantage using a regression by including a home 
indicator like the example above, and then using the hypothesis test where 
the null hypothesis is that the regression coefficient for home indicator
is zero. So it is important to, at minimum, discuss your choice to not use
a regression, and justify that
choice.
\end{comment}

 This is certainly a interesting suggestion and something we explored. However, we 
 elected to not add the regression to the paper. We did take the suggestion to add
 a paragraph to the paper explaining why the hypothesis testing alone sufficed to
 answering our research questions without need for the regression. We did not
 consider accounting for strength of opponent necessary. Given the study was 
 conducted using the playoff teams, match-ups were largely between strong teams 
 with small gaps in ability. Had the study been done using regular season play, 
 controlling for strength of opponent would have been beneficial since bottom-tier 
 teams would be included. More details on why the regression was not required are 
 included in the manuscript, in the final paragraph of the Methods section.

\begin{comment}
I’d be interested in hearing more thoughts about why away teams were helped,
but home teams weren’t hurt, in the bubble. You did mention “No longer having
to face the struggle of traveling, pressure from opposing fans, or playing on
an unfamiliar court”. A little more detail would be nice. Of the factors that
impact home advantage, which ones were unchanged for the home team, but
changed for the away team, and in what specific situation?
For example, when a team has two home games in a row, they didn’t travel 
pre-bubble in between those games, and still didn’t travel in the bubble. If 
their opponent, the away team, had 2 away games in a row, they did travel pre-bubble,
but didn’t have to travel in the bubble. So in that case there is a difference in home advantage.
But if the home team instead had an away game before their current home game,
then they had to travel before their home game, and a lack of travel advantage
for the home team doesn’t apply to that game. I would imagine this happens 
frequently, and that 30-40\% of home games are preceded by an away game.
Also, if away teams were impacted more, wouldn’t they also be better on 
defense, which would cause the home team’s score to go down, for example?
Do the home advantage only impact offensive performance?
\end{comment}

 Thank you for bringing this up, this comment allowed us to revisit the discussion 
 and derive a more clear well-thought out conclusion from the results. A few 
 paragraphs were added to the Discussion section(paragraphs 5 and 7) of the paper detailing why
 it is likely only away teams performance were significantly changed. The 
 discussion strengthened our determination that home-court advantage is best 
 categorized as negative effects on away team.

\begin{comment}
About study author-remove “will”
\end{comment}

Done.


\bibliographystyle{chicago}
\bibliography{citations}


\end{document}

%%% Local Variables:
%%% mode: latex
%%% TeX-master: t
%%% End:
